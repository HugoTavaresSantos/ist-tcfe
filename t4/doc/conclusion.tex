\section{Conclusion}
\label{sec:conclusion}

\paragraph{} In this laboratory assignment the objective of creating and analysing an Audio Amplifier has been achieved with success. 
We have performed theoreticall and simulation analysis, using the Octave for the former and Ngspice for the latter.

\paragraph{} We found some discrepancies between both sets of results, which can be atributed, among other things, the fact that the circuit 
does not start from equilibrium. The fact that the impedance was high made this more evident. While these discrepancies are not ideal, specially 
for real world aplications, they can be expected and were mitigated.

\paragraph{} The table bellow has the Theoretical and Simulation results, allowing for it's comparison.

\begin{table}[!h]
  \centering
  \begin{tabular}{c c c c}
    \hline    
    {\bf Theoretical} & {\bf Value} & {\bf Simulation} & {\bf Value}\\ \hline
    Frequency response and impedances &  &  &  \\ \hline
$Gain$ & 100.643363 & Gain & 99.7361\\ \hline
$Gain(dB)$ & 40.055703 dB & Gain(dB) & 39.977 dB\\ \hline
$LowerCut-offFreq$ & 723.431560 Hz & Lower cut-off freq & 403.611 Hz\\ \hline
$UpperCut-offFreq$ & 1446.863119 Hz & Upper cut-off freq & 2386.17 Hz\\ \hline
$CentralFreq$ & 1023.086723 Hz & Central freq & 981.37 Hz\\ \hline
$Z_{in} Modulus$ & 1234.241962 Ohm & Zin modulus & 1.23431 kOhm\\ \hline
$Z_{in} Phase$ & -35.883164 Degrees & Zin phase & -35.8894 Degrees\\ \hline
$Z_{out} Modulus$ & 822.637497 Ohm & Zout modulus & 0.826194 kOhm\\ \hline
$Z_{out} Phase$ & -34.650304 Degrees & Zout phase & -34.3981 Degrees\\ \hline
$Cost$ & 13626.952040 & Cost & 1.362695e+04\\ \hline
$Merit$ & 3.092445*$10^{-6}$ & Merit & 3.883972e-06\\ \hline
 
  \end{tabular}
  \caption{Comparison of the theoretical and simulated data results, regarding the operating point, frequency response and impedances.}
  \label{tab:comp}
\end{table}

\paragraph{} We also believe that, given the satisfactory results obtained by us, the model used could be applied in a real life Audio Amplifier.

\paragraph{} Finally, this assignment allowed us to gain some further knowledge in the application of the subjects topics.







