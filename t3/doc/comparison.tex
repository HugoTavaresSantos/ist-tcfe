\section{Results Comparison}
\label{sec:comparison}

As can be seen bellow, there are some differences between the NGSpice and Octave values for the average and ripple voltages of the envelope detector.


\begin{table}[h]
        \parbox{.45\linewidth}{
  \centering
  \begin{tabular}{|l|r|}
    \hline
    {\bf Name} & {\bf Value [V]} \\ \hline
    enveloperipple & 2.562498e-01 \\ \hline
envelopeaverage & 1.424920e+01 \\ \hline

  \end{tabular}
  \caption{Octave envelope ripple and average voltages.}
        \label{tab:envelopecomp}
}
\hfill
        \parbox{.45\linewidth}{
  \centering
  \begin{tabular}{|l|r|}
    \hline
    {\bf Name} & {\bf Value [V]} \\ \hline
    maximum(v(4))-minimum(v(4)) & 1.741797e-01\\ \hline
mean(v(4)) & 1.276542e+01\\ \hline

  \end{tabular}
  \caption{NGSpice envelope ripple and average voltages.}
  \label{tab:envcomp}
}
\end{table}



These differences relate to the fact that the diodes used are non-linear components. This means that any linear relations established between currents and voltages are approximations of the real circuit, which contributes to some discrepancies between results.

Despite that, these discrepancies are expected, justified and small, which leads us to conclude that this model for the envelope detector is valid and yields a high precision.  

In a similar manner, some differences between the NGSpice and Octave values for the average and ripple voltages of the voltage regulator were found, as can be seen bellow.


\begin{table}[h]
        \parbox{.45\linewidth}{
  \centering
  \begin{tabular}{|l|r|}
    \hline
    {\bf Name} & {\bf Value [V]} \\ \hline
    regulatorripple & 6.729519e-02 \\ \hline
regulatoraverage & 1.200000e+01 \\ \hline

  \end{tabular}
  \caption{Octave regulator ripple and average voltages.}      
        \label{tab:regulatorcomp}
}
\hfill
        \parbox{.45\linewidth}{
  \centering
  \begin{tabular}{|l|r|}
    \hline
    {\bf Name} & {\bf Value [V]} \\ \hline
    maximum(v(5))-minimum(v(5)) & 1.933408e-02\\ \hline
mean(v(5)) & 1.331470e+01\\ \hline
     
  \end{tabular}
  \caption{NGSpice regulator ripple and average voltages.}
  \label{tab:regcomp}
}
\end{table}




These discrepancies can be explained by the same aforementioned reasons, namely the non-linear behavior of the diodes used. However, we can conclude that the model used was successful, because it achieved the main goal of producing an output voltage of approximately 12V. 

Finally, the cost and merit of this circuit can be found in Table~\ref{tab:merit}.


\begin{table}[ht]
  \centering
  \begin{tabular}{|l|r|}
    \hline    
    {\bf Name} & {\bf Value} \\ \hline
        1/ (515.8* ((maximum(v(5))-minimum(v(5))) + abs(mean(v(5)-12)) + 10e-6)) & 7.529896e-04\\ \hline

  \end{tabular}
  \caption{Merit figure} 
        \label{tab:merit}
\end{table}     










