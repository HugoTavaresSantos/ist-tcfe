\section{Conclusion}
\label{sec:conclusion}
\par
\begin{table}[!h]
  \centering
  \begin{tabular}{c c c c}
    \hline    
    {\bf Theoretical} & {\bf Value} & {\bf Simulation} & {\bf Value}\\ \hline
    \input{comp.tex} 
  \end{tabular}
  \caption{Comparison of the theoretical and simulated data results, regarding the frequency response and impedances.}
  \label{tab:comp}
\end{table}

\paragraph{} In this laboratory assignment, we managed to build a BPF circuit with success. The first step of our analysis was to determine the frequency response by computing
 the transfer function of the whole circuit.

\paragraph{} In the previous report we noted that there were some discrepancies between the theoretical and simulation results in the transistor. In this lab we can expect an agravation of this, given that we are dealing
with a high number of transostors. In other words, this can couse the theoretical and simulation results to differ significantly. Also, the complexity of the OPAMP model used in the Ngspice and the parasitic 
capacity may have played a role in the results obtained.

\paragraph{} Due to this, the theoretical and simulation results differ significantly in some aspects. 

\paragraph{} We also believe that, given the somewhat satisfactory results obtained by us, the model used could be applied in a real life BPF circuit.

\paragraph{} Finally, this assignment allowed us to gain some further knowledge in the application of the subjects topics.







