\section{Conclusion}
\label{sec:conclusion}
\par
\begin{table}[!h]
  \centering
  \begin{tabular}{c c c c}
    \hline    
    {\bf Theoretical} & {\bf Value} & {\bf Simulation} & {\bf Value}\\ \hline
    Frequency response and impedances &  &  &  \\ \hline
$Gain$ & 100.643363 & Gain & 99.7361\\ \hline
$Gain(dB)$ & 40.055703 dB & Gain(dB) & 39.977 dB\\ \hline
$LowerCut-offFreq$ & 723.431560 Hz & Lower cut-off freq & 403.611 Hz\\ \hline
$UpperCut-offFreq$ & 1446.863119 Hz & Upper cut-off freq & 2386.17 Hz\\ \hline
$CentralFreq$ & 1023.086723 Hz & Central freq & 981.37 Hz\\ \hline
$Z_{in} Modulus$ & 1234.241962 Ohm & Zin modulus & 1.23431 kOhm\\ \hline
$Z_{in} Phase$ & -35.883164 Degrees & Zin phase & -35.8894 Degrees\\ \hline
$Z_{out} Modulus$ & 822.637497 Ohm & Zout modulus & 0.826194 kOhm\\ \hline
$Z_{out} Phase$ & -34.650304 Degrees & Zout phase & -34.3981 Degrees\\ \hline
$Cost$ & 13626.952040 & Cost & 1.362695e+04\\ \hline
$Merit$ & 3.092445*$10^{-6}$ & Merit & 3.883972e-06\\ \hline
 
  \end{tabular}
  \caption{Comparison of the theoretical and simulated data results, regarding the frequency response and impedances.}
  \label{tab:comp}
\end{table}

In this laboratory assignment, we managed to build a BandPass Filter (BPF) circuit which is represented in Figure 1. The first step of our analysis was to determine the frequency response by computing the transfer function of the whole circuit.

In the last report we explained how the theoretical model of the transistor lacked the complexity needed to yield closer results to the simulation. In this assignment we are dealing with 22 separate transistors inside the OP-AMP. This alone means that theoretical model used to analyse the circuit can differ significantly since it isn't expected to take into account the non-linearity of transistors. Adding to that is the complexity of the OP-AMP model used in Ngspice, especially the use of various capacitors and diodes which weren't taken into account in the octave analysis. Furthermore, another thing that was not taken into account was the parasitic capacitance of the 22 transistors inside de OP-AMP. 

Due to this, the theoretical and simulation results differ significantly in some aspects. 
The bandwidth, despite still being approximately centered around the 1 KHz frequency, was much bigger (double the width).We've seen in the previous report that this might have been a source of error because we were not able to get a value for the upper cut off frequency. We believe that the issue with the bandwith in the present case might have had the parasitic capacitance of the transistors as a source.
%não sei se isto faz muito sentido

%This is due mainly (as would be expected by now) to the fact that the OPAMP model used includes capacitors, which introduces, by means of additional complex impedances, two extra poles in the transfer function. Not only does this subvert the format for the phase bode plot obtained theoretically, it also affects the upper cutoff frequency, and, as a result, the central passband frequency: while, in theory, the parameters used will not give us the desired frequency (in fact it would be around 100Hz and not around 1kHz), in practice, they do! \par

Despite all of this, the merit figures are somewhat similar, especially if we consider the significant differences from previous assignments.








