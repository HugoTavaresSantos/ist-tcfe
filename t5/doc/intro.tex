
\section{Introduction}
\label{sec:introduction}


\paragraph The objective of this laboratory assignment is to build a BandPass Filter (BPF), this is, a circuit that allow frequencies with a certain range to pass but attenuate all others.
The specifications for this lab project are: a central frequency at 1000Hz and a gain at central frequency of 40dB.
 
Our circuit is comprised of a $\mu$A741 OPAMP connected to two resistors, creating a non-inverting amplifier, and a combination of resistors and capacitors to take care of the filtering, as it can 
be seen in figure 1.

The signal goes through a first stage where we have a capacitor whose function is to block unwanted DC current and to filter out lower frequencies, hence this stage being a high pass filter, 
since the rest of the circuit is connected to the terminals of the resistor. Now the signal goes through a combination of resistors and a $\mu$741 OPAMP. This arrangement creates 
a non-inverting amplifier. In this configuration, the output signal is "in-phase" with the input signal. Feedback control of the non-inverting OPAMP is achieved by applying a small part of 
the output voltage back to the inverting (-) terminal via R3-R4 voltage divider network. And finally, the signal is subjected to a voltage divider network comprised of resistor R2 and capacitor C2. 
The desired voltage corresponds to the voltage drop at the terminals of the capacitor which, as a result, is subjected to a low pass filtering. For this to run, we need two supply DC voltage sources 
overlapped with our main circuit in order to power the transistors inside the OPAMP.

\begin{figure}[h] \centering
\includegraphics[width=0.9\linewidth]{circuit.pdf}
\caption{Geometry of our Band-Pass filter circuit.}
\end{figure}

This report is subdivided into the following sections: Section 2, in which we layed out the theororical models and calculations used to determine the transfer fucntion and, consequentilly, the frequency response.
In Section 3 we display the results obtained in the simulation. Finally, in Section 4, we will compare both the theoretical results and draw conclusions.

In the table below we display the values associated with each component used:

\begin{table}[h]
  \centering
  \begin{tabular}{|l|r|}
    \hline    
    {\bf Name} & {\bf Values} \\ \hline
    $R_1$ & 1.000000 kOhm \\ \hline 
$C_1$ & 0.220000 uFarad \\ \hline 
$R_2$ & 1.000000 kOhm \\ \hline 
$C_{21}$ & 0.220000 uFarad \\ \hline 
$C_{22}$ & 0.220000 uFarad \\ \hline 
$R_3$ & 1.000000 kOhm \\ \hline 
$R_{4a}$ & 100.000000 kOhm \\ \hline 
$R_{4b1}$ & 100.000000 kOhm \\ \hline 
$R_{4b2}$ & 100.000000 kOhm \\ \hline 
 
  \end{tabular}
  \caption{Values of components used in our analysis and simulation.}
  \label{tab:data}
\end{table}
