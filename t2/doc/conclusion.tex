\section{Conclusion}
\label{sec:conclusion}

\paragraph{} In this laboratory assignment we analysed the circuit indicated in Figure 1 successfully. This analysis was done both theoretically, making use of the 

The theoretical estimates were compared with the simulation results, yielding very similar results. The plots obtainted from performing the nodal method in GNU Octave 
and the simulation in NGSpice are mostly identical, with errors being of such low order of magnitude that they become negligible, and are most likely attributable to rounding errors. 
This outcome was expected, since the circuit is composed of linear components, and hence NGSpice likely used the same methods as Octave did in the theoretical part. Because the 
results are coherent and within expectation, we consider this task to be successful.

Kirchoff's Laws
 and the Octave math tool to perform the calculations, and by circuit simulation, using the Ngspice tool. Unlike in theory, in real life we could expect several experimental 
errors, due to the measurement instruments, the way the measurement is performed, internal resistance of the components of the circuit, temperature, etc.

However, using the computer simulation, we came to the conclusion that the results we obtained matched the ones we calculated theoretically precisely.
This is because this is a relatively simple circuit with few components. This corroborates with what was taught in theory classes, that theoretical and simulation models
 cannot differ; unless we are talking about more complex circuits, which is clearly not the case.

Finally, this lab assignment was useful to put to test our knowledge of circuits and the "laws" that they obey to and familiarize ourselves with sophisticated softwares,
 like Git, Makefile, Octave and Ngspice.